% Options for packages loaded elsewhere
\PassOptionsToPackage{unicode}{hyperref}
\PassOptionsToPackage{hyphens}{url}
\PassOptionsToPackage{dvipsnames,svgnames,x11names}{xcolor}
%
\documentclass[
]{agujournal2019}

\usepackage{amsmath,amssymb}
\usepackage{iftex}
\ifPDFTeX
  \usepackage[T1]{fontenc}
  \usepackage[utf8]{inputenc}
  \usepackage{textcomp} % provide euro and other symbols
\else % if luatex or xetex
  \usepackage{unicode-math}
  \defaultfontfeatures{Scale=MatchLowercase}
  \defaultfontfeatures[\rmfamily]{Ligatures=TeX,Scale=1}
\fi
\usepackage{lmodern}
\ifPDFTeX\else  
    % xetex/luatex font selection
\fi
% Use upquote if available, for straight quotes in verbatim environments
\IfFileExists{upquote.sty}{\usepackage{upquote}}{}
\IfFileExists{microtype.sty}{% use microtype if available
  \usepackage[]{microtype}
  \UseMicrotypeSet[protrusion]{basicmath} % disable protrusion for tt fonts
}{}
\makeatletter
\@ifundefined{KOMAClassName}{% if non-KOMA class
  \IfFileExists{parskip.sty}{%
    \usepackage{parskip}
  }{% else
    \setlength{\parindent}{0pt}
    \setlength{\parskip}{6pt plus 2pt minus 1pt}}
}{% if KOMA class
  \KOMAoptions{parskip=half}}
\makeatother
\usepackage{xcolor}
\setlength{\emergencystretch}{3em} % prevent overfull lines
\setcounter{secnumdepth}{5}
% Make \paragraph and \subparagraph free-standing
\makeatletter
\ifx\paragraph\undefined\else
  \let\oldparagraph\paragraph
  \renewcommand{\paragraph}{
    \@ifstar
      \xxxParagraphStar
      \xxxParagraphNoStar
  }
  \newcommand{\xxxParagraphStar}[1]{\oldparagraph*{#1}\mbox{}}
  \newcommand{\xxxParagraphNoStar}[1]{\oldparagraph{#1}\mbox{}}
\fi
\ifx\subparagraph\undefined\else
  \let\oldsubparagraph\subparagraph
  \renewcommand{\subparagraph}{
    \@ifstar
      \xxxSubParagraphStar
      \xxxSubParagraphNoStar
  }
  \newcommand{\xxxSubParagraphStar}[1]{\oldsubparagraph*{#1}\mbox{}}
  \newcommand{\xxxSubParagraphNoStar}[1]{\oldsubparagraph{#1}\mbox{}}
\fi
\makeatother


\providecommand{\tightlist}{%
  \setlength{\itemsep}{0pt}\setlength{\parskip}{0pt}}\usepackage{longtable,booktabs,array}
\usepackage{calc} % for calculating minipage widths
% Correct order of tables after \paragraph or \subparagraph
\usepackage{etoolbox}
\makeatletter
\patchcmd\longtable{\par}{\if@noskipsec\mbox{}\fi\par}{}{}
\makeatother
% Allow footnotes in longtable head/foot
\IfFileExists{footnotehyper.sty}{\usepackage{footnotehyper}}{\usepackage{footnote}}
\makesavenoteenv{longtable}
\usepackage{graphicx}
\makeatletter
\newsavebox\pandoc@box
\newcommand*\pandocbounded[1]{% scales image to fit in text height/width
  \sbox\pandoc@box{#1}%
  \Gscale@div\@tempa{\textheight}{\dimexpr\ht\pandoc@box+\dp\pandoc@box\relax}%
  \Gscale@div\@tempb{\linewidth}{\wd\pandoc@box}%
  \ifdim\@tempb\p@<\@tempa\p@\let\@tempa\@tempb\fi% select the smaller of both
  \ifdim\@tempa\p@<\p@\scalebox{\@tempa}{\usebox\pandoc@box}%
  \else\usebox{\pandoc@box}%
  \fi%
}
% Set default figure placement to htbp
\def\fps@figure{htbp}
\makeatother

\usepackage{url} %this package should fix any errors with URLs in refs.
\usepackage{lineno}
\usepackage[inline]{trackchanges} %for better track changes. finalnew option will compile document with changes incorporated.
\usepackage{soul}
\linenumbers
\makeatletter
\@ifpackageloaded{caption}{}{\usepackage{caption}}
\AtBeginDocument{%
\ifdefined\contentsname
  \renewcommand*\contentsname{Table of contents}
\else
  \newcommand\contentsname{Table of contents}
\fi
\ifdefined\listfigurename
  \renewcommand*\listfigurename{List of Figures}
\else
  \newcommand\listfigurename{List of Figures}
\fi
\ifdefined\listtablename
  \renewcommand*\listtablename{List of Tables}
\else
  \newcommand\listtablename{List of Tables}
\fi
\ifdefined\figurename
  \renewcommand*\figurename{Figure}
\else
  \newcommand\figurename{Figure}
\fi
\ifdefined\tablename
  \renewcommand*\tablename{Table}
\else
  \newcommand\tablename{Table}
\fi
}
\@ifpackageloaded{float}{}{\usepackage{float}}
\floatstyle{ruled}
\@ifundefined{c@chapter}{\newfloat{codelisting}{h}{lop}}{\newfloat{codelisting}{h}{lop}[chapter]}
\floatname{codelisting}{Listing}
\newcommand*\listoflistings{\listof{codelisting}{List of Listings}}
\makeatother
\makeatletter
\makeatother
\makeatletter
\@ifpackageloaded{caption}{}{\usepackage{caption}}
\@ifpackageloaded{subcaption}{}{\usepackage{subcaption}}
\makeatother

\usepackage{bookmark}

\IfFileExists{xurl.sty}{\usepackage{xurl}}{} % add URL line breaks if available
\urlstyle{same} % disable monospaced font for URLs
\hypersetup{
  pdftitle={Яка машиночитаність потрібна для доступності використання відкритих даних},
  pdfauthor={Микола Кузін},
  pdfkeywords={машиночитаність, відкриті дані, стандарти відкритих
даних},
  colorlinks=true,
  linkcolor={blue},
  filecolor={Maroon},
  citecolor={Blue},
  urlcolor={Blue},
  pdfcreator={LaTeX via pandoc}}



\draftfalse

\begin{document}
\title{Яка машиночитаність потрібна для доступності використання
відкритих даних}

\authors{Микола Кузін\affil{1}}
\affiliation{1}{BRDO, }
\correspondingauthor{Микола Кузін}{m.kuzin@brdo.com.ua}


\begin{abstract}
Машиночитаність --- це структурованість документа на логічному рівні, що
уможливлює автоматизоване зчитування його структури та змісту. У сфері
відкритих даних цей термін важливо розглядати разом з поняттями
``доступності використання'' й ``простоти обробки'' наборів даних ---
тобто приймати до уваги користувацьку перспективу. З точки зору
користувача важлива не лише публікація набору у машиночитаному
(структурованому) форматі, а й дотримання стандартів публікації даних. В
цій статті наводжу аргументи, чому для українських розпорядників назагал
важливішесфокусуватися на користувацькій перспективі й усунути бар'єри
використання наборів, які вже публікуються --- й лише після цього (за
потреби) дивитися в бік стандартів W3C і Інтернету Даних.
\end{abstract}





\section{Відкриті дані: вільний доступ і доступність
використання}\label{ux432ux456ux434ux43aux440ux438ux442ux456-ux434ux430ux43dux456-ux432ux456ux43bux44cux43dux438ux439-ux434ux43eux441ux442ux443ux43f-ux456-ux434ux43eux441ux442ux443ux43fux43dux456ux441ux442ux44c-ux432ux438ux43aux43eux440ux438ux441ux442ux430ux43dux43dux44f}

За визначенням Open Knowledge Foundation ---розробників CKAN, на якій
реалізований український Портал відкритих даних --- ``відкритість''
даних полягає в тому, що будь-хто може мати до них вільний доступ,
вільно використовувати, змінювати та ділитися ними\footnote{\url{https://opendefinition.org/od/2.0/ua/}}.

\href{https://aims.gitbook.io/open-data-mooc/unit-1-open-data-principles/lesson-1.1-what-is-open-data\#id-5.-challenges}{Однак
``вільний'' не завжди значить ``відкритий''}: часто потрібні додаткові
кроки, щоби з інформації у вільному доступі зробити дані, доступні до
використання.

Доступність використання --- важлива категорія. У Постанові КМУ № 835
принцип ``доступності використання'' відкритих даних напряму
пов'язується з машиночитаним форматом оприлюднених даних. А
машиночитаність --- зі структурованістю, що уможливлює обробку без
участі людини (власне, машинну обробку). У тому ж документі визначено
перелік структурованих форматів, що використовуються для оприлюднення
наборів. Сюди відносяться \texttt{.csv}, \texttt{.xls(x)},
\texttt{.json}, \texttt{.rdf} та інші.

Але візьмемо набір даних
``\href{https://data.gov.ua/dataset/8b9b1677-2407-454a-bfa7-76eb638c0ea1}{Єдиний
звіт про кримінальні правопорушення}'', що оприлюднюється Генеральною
Прокуратурою на Порталі відкритих даних. Цей набір даних оприлюднюється
у структурованому форматі \texttt{.xlsx} і відтак, відповідно до
визначень вище, відповідає принципу доступності використання. Втім,
заглянуваши всередину опублікованих ресурсів (файлів) набору, побачимо,
чому про ``машиночитаність'' та ``доступність використання'' тут можна
говорити досить умовно:

\begin{itemize}
\tightlist
\item
  Всередині однієї вкладки можуть розміщуватися кілька таблиць, одна під
  одною. Розрізнити їх між собою для роботи в середовищі розробки ``без
  участі людини'' складно.
\item
  Самі таблиці використовуються для відображення ієрархічних структур
  даних: для прикладу, за рядком ``Усього кримінальних правопорушень''
  слідують рядки з різнорівневими значеннями ``з них''. Екселівські
  таблиці --- не кращий спосіб відображення даних, що мають ієрархічну
  структуру.
\item
  Назви колонок йдуть у кілька рядків, частина цих клітинок злита між
  собою, що далі ускладнює зчитування цих таблиць у середовище розробки.
\item
  Значення колонок сумуються --- це різновид дублювання даних і такі
  рядки треба додатково вичищати.
\end{itemize}

На перший погляд маємо суперечність: набір даних опублікований у
структурованому форматі \texttt{.xlsx} і тому є машиночитаним, але для
прочитання в середовищі розробки (для машинної обробки) потребує
суттєвої участі людини і тому не є машиночитаним.

У Хартії відкритих даних, що описує шість основоположних принципів
публікації наборів у форматі відкритих даних, ця суперечність знімається
за умови дотримання третього та четвертого принципів. В одному
наголошується на важливості використовувати структуровані (машиночитані)
формати, тоді як інший вказує на необхідність дотримуватися ``поширених
стандартів'' публікації даних.

Перед тим як детальніше сказати про користувацьку перспективу і
стандарти публікації даних, розглянемо досить спеціальний контекст
використання поняття машиночитаності --- а саме стандарти публікації
даних W3C. Оскільки і ``стандарти публікації даних W3C'' і ``поширені
стандарти публікації даних'' містять слово ``стандарти'', а тексти з
рекомендаціями стосовно кожного з них містяться на Порталі відкритих
даних, треба чітко відділити одне від одного.

\section{\texorpdfstring{\textbf{Стандарти W3C і машиночитаність як
інтегрованість набору в Інтернет
Даних}}{Стандарти W3C і машиночитаність як інтегрованість набору в Інтернет Даних}}\label{ux441ux442ux430ux43dux434ux430ux440ux442ux438-w3c-ux456-ux43cux430ux448ux438ux43dux43eux447ux438ux442ux430ux43dux456ux441ux442ux44c-ux44fux43a-ux456ux43dux442ux435ux433ux440ux43eux432ux430ux43dux456ux441ux442ux44c-ux43dux430ux431ux43eux440ux443-ux432-ux456ux43dux442ux435ux440ux43dux435ux442-ux434ux430ux43dux438ux445}

Винахідник мережі Інтернет Тім Бернерс-Лі розповідав, що на створення
Всесвітньої Мережі його наштовхнули постійні складнощі обміну
документами між різними користувачами. Оскільки це були в основному
текстові документи (те, що можна прочитати), звідти маємо
``гіпер-текстові'' (hypertext, HT) посилання, HTTP-протокол,
HTML-розмітку для структурування веб-сторінок і ще ряд засадничих для
сьогоднішньої Мережі стандартів та технологій.

Згодом він написав статтю\footnote{\url{https://www.w3.org/DesignIssues/LinkedData.html}},
де заявив, що поряд із гіпертекстовою мережею (мережею документів),
потрібна мережа даних, оскільки даних стає все більше --- і всі
виграють, якщо набори даних будуть пов'язані з іншими релевантними
наборами по всій Всесвітній Мережі. Він сформулював поняття Linked Data,
``пов'язаних {[}лінками{]} даних'', або ``залінкованих даних''. На
відміну від гіпертекстової мережі, де лінки (посилання) на інші
документи містяться в тезі \texttt{\textless{}a\textgreater{}} з
атрибутом \texttt{href} у HTML-документі, об'єкт мережі даних містить
посилання на інші об'єкти в RDF-документі.

Так, якщо розпорядник в Україні, скажімо, КМДА, публікує набір даних про
парковки в Київі, при дотриманні стандарту RDF (зокрема, використанні
словників DCAT), цей набір буде залінкований з усіма наборами у світі, в
яких йдеться про парковки в містах --- якщо в них дотримані ті ж
стандарти. RDF використовує троїсті структури (триплети) для опису
спостережень:

\begin{enumerate}
\def\labelenumi{\arabic{enumi}.}
\tightlist
\item
  Тут буде \texttt{суб\textquotesingle{}єкт}, конкретне паркомісце в
  Києві зі своїм унікальним на весь Інтернет URI-ідентифікатором;
\item
  \texttt{предикат}, який описуватиме властивість цього суб'єкта ---
  наприклад, ``Кількість паркомісць'';
\item
  \texttt{об\textquotesingle{}єкт} --- це значення предиката. Скажімо,
  ``70'' (паркомісць).
\end{enumerate}

Цей гарний інтерактивний граф з Linked Open Data Cloud показує, яка
виглядає Інтернет даних, завдяки організаціям, які дотримуються
стандартів відкритих даних WC3:

Я навмисно тут дещо заглибився в технічні деталі, щоб показати, що
робота з RDF-форматом --- це не просто\footnote{Це лагідний вступ до
  роботи з RDF (на мові R)
  \url{https://cran.r-project.org/web/packages/rdflib/vignettes/rdf_intro.html}}.
Для повноцінного використання стандартів відкритих даних W3C
розпорядникам потрібні спеціалісти. На українському Порталі відкритих
даних станом на 11 грудня 2024 року з 841,643 опублікованих ресурсів
(файлів), у форматі RDF опубліковано лише 25 (двадцять п'ять). При чому
усі у 2018 році і в одного розпорядника, Державної прикордонної служби.

\begin{verbatim}
# A tibble: 25 x 2
   mimetype            created                   
   <chr>               <chr>                     
 1 application/rdf+xml 2018-11-01T13:28:03.538971
 2 application/rdf+xml 2018-11-02T09:00:38.201840
 3 application/rdf+xml 2018-11-01T11:14:31.685839
 4 application/rdf+xml 2018-11-05T10:21:52.515944
 5 application/rdf+xml 2018-11-05T10:37:34.390529
 6 application/rdf+xml 2018-11-05T10:39:31.313447
 7 application/rdf+xml 2018-11-02T11:02:25.354277
 8 application/rdf+xml 2018-11-05T10:22:31.868074
 9 application/rdf+xml 2018-11-05T10:41:03.930371
10 application/rdf+xml 2018-11-05T10:23:24.421427
# i 15 more rows
\end{verbatim}

\textsubscript{Source:
\href{https://mykuzin.github.io/machine-readability-and-open-data/index.qmd.html}{Article
Notebook}}

\section{Поширені стандарти публікації даних і увага до
користувачів}\label{sec-data-methods}

Перше, що кинулося мені у вічі, коли почав читати обговорення\footnote{\url{https://github.com/co-cddo/open-standards/issues/40}}
спільноти відкритих даних Сполученого королівства --- це уточення
\emph{user stories}, тобто сценаріїв використання наборів даних різними
групами користувачів. Цей же підхід є у рекомендованих країною
стандартах публікації відкритих даних. Для прикладу, є рекомендований
стандарт публікації табличних даних\footnote{\url{https://www.gov.uk/government/publications/recommended-open-standards-for-government/tabular-data-standard}},
в якому на самому початку описані користувачі таких даних:

\begin{itemize}
\tightlist
\item
  люди, які використовують табличний редактор для базового аналізу
\item
  аналітики, які використовують дані в статистичних програмах або
  застосунках для бізнес-аналітики для проведення інтерактивного аналізу
\item
  дата-сайєнтисти, які пишуть програмне забезпечення для аналізу даних,
  яке завантажує та обробляє дані, наприклад, відтворювані аналітичні
  пайплайни
\item
  розробники, які обробляють дані в різному програмному забезпеченні
\item
  люди, яким потрібно швидко шукати релевантні дані перед їх аналізом за
  допомогою спеціалізованих інструментів
\end{itemize}

Відповідно, знаючи профілі тих, хто користуватиметься даними, стають
зрозумілішіми їхні потреби, з яких вже можна сформулювати вимоги до
публікації самих табличних наборів. За такого підходу доступність й
простота використання справді стоїть на першому місці.

На противагу, повертаючись до згаданого набору Генпрокуратури, єдиною
групою користувачів, яку можна уявити розглядаючи набір у його поточному
стані --- це люди, які користуються роздрукуваними листками. Втім, сам
факт публікації цього набору розпорядником є важливою віхою; питання
стоїть як цей набір і решту табличних наборів зробити зручнішими для
використання.

В Україні є сформульовані рекомендації щодо публікації табличних даних,
що роблять простішою для користувачів роботу з ними. У пам'ятці
«Підготовка даних до публікації», розробленої в рамках проекту
«Прозорість та підзвітність в державному управлінні та послугах» /
TAPAS, влучно зазначено, що «\ldots{} найбільш цінними для користувачів
є саме структуровані та машиночитані дані. Однак із цим типом даних
традиційно виникає найбільше проблем у розпорядників даних»\footnote{\url{https://data.gov.ua/uploads/files/2018-08-11-104337.710875Part04.pdf}}.
Там же наведені конкретні приклади, як виглядають правильні і
неправильні таблиці --- і чому так.

Крім того, на Порталі відкритих даних є лаконічні рекомендації з
принципами оприлюднення табличних даних (вони потребують уточнення і
розширення, щоб справді бути корисними розпорядникам):

\begin{itemize}
\tightlist
\item
  Таблиця --- це впорядкована сукупність колонок та рядків.
\item
  Кожен рядок таблиці містить один запис.
\item
  Кожна колонка --- значення, що змінюються від рядка до рядка.
\item
  Назви колонок розміщуються в шапці. \# \emph{тут треба ще додати, що в
  одному рядку}
\item
  На перетині рядків та колонок знаходяться комірки.
\item
  Таблиця не може містити заголовків та об'єднаних комірок.
\item
  Колір, шрифт, інше форматування тексту та комірок не є даними.
\end{itemize}

Таким чином, в Україні проблема простоти та доступності використання
структурованих даних, як бачимо, поставлена, поточна мета --- засвоєння
базових рекомендацій щодо публікацій наборів якомога ширшим колом
розпорядників.

\section{Висновки}\label{ux432ux438ux441ux43dux43eux432ux43aux438}

Візіонерський проєкт Інтернету даних, описаний у стандартах відкритих
даних W3C, вимагає від ропорядників ресурсів та інституційної
спроможності для реалізації цих проєктів. Більшість ЦОВВ і окремі ОМС в
Україні можуть це робити --- й отримають більше видимості у світі та
інтеграції з релевантними для них стейкхолдерами.

Втім, є прагматичніша ціль, від якої виграють усі користувачі даних в
Україні --- привести ті набори даних, що вже публікуються, у
відповідність базовим рекомендаціям (стандартам) до публікації наборів.




\end{document}
